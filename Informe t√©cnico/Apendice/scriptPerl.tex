\chapter{\textsc{ Script Perl }}\label{scriptPerl}
%\lstset{language=Perl}
\small{
\begin{verbatim}
#!/usr/bin/perl 

#### Import Classes
use File::Copy;
use FileHandle;
####
#### Define constants
my $idPort = "id";

# Input File
my $file =$ARGV[0];

#
my %ports;	# to store ports (ins & outs)
		# and wire's name given
		# by the VhdlNew.hs

### OPEN INPUT FILE 
print "$file\n";
open(my $fhi, '<',$file) or die "archivo no encontrado";
open(my $fho, '+>',"temp-$file");
while(<$fhi>){
#Primero obtengo las entradas
if(m/.+$idPort.+port.map.+\(([A-Za-z0-9_]+).+(w[0-9]+)/)
	{
	push @wires,$2;
	push @wires,$1;
	$ports {$2} = $1;
#	print $fho "--$_"; # comento la linea
	} 

#Ahora obtengo las salidas, por ejemplo:
# c_sum_0   :  std_wire  port map (w1, sum_0);
# o como estas:
# c_cout    :  std_wire  port map (w131, cout);
if(m/.+$idPort.+(w[0-9]+)\,.?([A-Za-z0-9_]+)\).*/)                          
{
$ports {$1} = $2;
print $fho "--$_"; # comento lo que quiero eliminar
} else {
print $fho "$_";}
# and the outputs
#if(m/([a-z]+_[a-z]*_*[0-9]+).+out/) {push @outs,$1;}
# I could use ins and outs to make the %ports hash table
		}#while
close($fhi);
close($fho);


#copy("temp-$file", "temp2-$file");
# Replace all signals connected to wire with the inputs 
#   c_w131    :  std_or2   port map (w132, w141, w131);
# w131 should be replaced by the output


while(my($key,$value) = each(%ports)) {
	open(my $fhi, '<',"temp-$file") or die "archivo no encontrado";
	open(my $fho, '+>', "stripped-$file");
	while(<$fhi>){
		if(s/(.+map.+)$key(\,|\))/$1$value$2/g) {

#need to delete entries with the next pattern:  c_w18     :  std_wire  port map ...
			if(m/.+$idPort.+/) {print $fho "-- deleted $_";}
			else {print $fho "$_";}}
		else { print $fho "$_";}
		      }# while file 

	close($fho);
	close($fhi); #atenti no hacer close($fho, $fhi) porque no es lo mismo que en 2 renglones
	copy("stripped-$file","temp-$file");
					}#while hash table
\end{verbatim}
} %cierra el \small
