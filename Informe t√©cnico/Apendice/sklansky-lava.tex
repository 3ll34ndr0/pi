\chapter{\textsc{ Descripcion en Lava de los sumadores Sklansky y Brent-Kung}}\label{chap:sklansky-lava}
\section{Sumador de Sklansky}
\lstset{language=Haskell} 
\begin{lstlisting}
sklansky :: [(Bit, Bit)] -> ([Bit], Bit)
sklansky abs = (ss,cout)
  where
    gps            = map gpC abs
    (cs,cout)      = (skl (mkFan dotOp) ->- unsnoc ->- (map fst -|- fst) ) gps
    ((_,p) : gps') = gps
    rs             = zip cs gps'
    ss             = p : map sumC rs
--
-- 
{-- La funcion dotOp utilizada es la
 misma que la que definimos para Brent-Kung.

dotOp ((g1, p1) ,(g, p)) = (go, po)
   where
      go = or2 (g, and2 (p, g1))
      po = and2 (p, p1)
--}

{--
Funciones auxiliares utilizadas en el top level de sklanksy:
--}
gpC:: (Bit,Bit) -> (Bit,Bit) -- Genera los g y p a partir de los bit de entrada a y b.
gpC (a,b) = (a <&> b,a <#> b)

sumC :: (Bit,(Bit,Bit)) -> Bit -- Realiza la ultima operacion (una XOR ) para calcular la suma
sumC (cin, (_,p)) = cin <#> p

mkFan :: ((a,a) -> a) -> Fan a
mkFan op (i:is) = i:[op(i,k) | k <- is]

fsT f = (f  -|- id)
snD f = (id -|-  f)

unsnoc as = (init as, last as)
--

-- Red de prefijo de sklansky:
skl :: PP a
skl _ [a] = [a]
skl f as  = init los ++ ros'
  where
    (los,ros) = (skl f las, skl f ras)
    ros'      = f (last los : ros)
    (las,ras) = splitAt (cnd2 (length as)) as

cnd2 n = n - n `div` 2 -- El techo de n/2
--

{--
 Version del sumador que acomoda el tipo de datos de sklansky
 para hacerlo compatible con el modulo para generar el netlist VHDL.
 La version definida arriba toma una lista de tuplas. Y la que 
 creamos aqui toma una tupla de listas:
--}
sklansky_ :: ([Bit], [Bit]) -> ([Bit], Bit) 
sklansky_ (as,bs) = sklansky (zip as bs)
--

\end{lstlisting}


\section{Sumador de Brent-Kung}

\begin{lstlisting}
dotOp ((g1, p1) ,(g, p)) = (go, po)
   where
      go = or2 (g, and2 (p, g1))
      po = and2 (p, p1)
--
unzipl []        = ([],[])
unzipl [a]       = ([a], [])
unzipl (a:b:abs) = (a:as, b:bs)
   where
      (as, bs) = unzipl abs
--
zipl ([], [])     = []
zipl ([a], [])    = []
zipl (a:as, b:bs) = a:b:zipl(as, bs)
--
buf (gin,pin) = (gin, pin)
{-- 
dop toma un operador y un buffer y hace una funcion de 
una lista de duplas a lista de duplas.
Esto es lo que vamos a usar como red de 
prefijo cuando tenemos dos entradas
--}
dop [a,b] = [a, dotOp(a,b)]
--

miti p = unzipl ->- (id -|- p) ->- zipl
--

comb []     = []
comb [a]    = []
comb (a:as) = dop [a, head as]++comb (tail as)
--

posComb (a:as)  = a: (comb (init as))++ [last as]
--
wrapR p = comb ->- miti p ->- posComb 
 
{--
Toda la version del sumador está hecha para que no
 pida como parámetro el dotOp, con esta version no
 se puede usar el programa de Mary para hacer los 
dibujos de las redes, además de perder de vista 
donde se puede poner un buffer real. Esto no es
un gran problema ya que las herramientas de
place & route tienen la capacidad de agregar
o quitar buffers segun sea necesario.
--}
bKung [a] = []
bKung [a, b] = dop [a, b]
bKung as  = wrapR (bKung) as

--
gAndPs ([],[]) = []
gAndPs (a:as, b:bs) = (g,p):gps
   where
      (g, p) = (and2 (a, b),xor2 (a, b))
      gps    = gAndPs (as, bs)
--

fork as = (as, as)
--

evens :: [(Signal Bool,Signal Bool)] -> [Signal Bool]
evens as = cs
   where
      (bs,cs) = unzipp as
--

odds ::  [(Signal Bool,Signal Bool)] -> [Signal Bool]
odds as = bs
   where
      (bs,cs) = unzipp as

--

dropP :: ([Signal Bool], [(Signal Bool,Signal Bool)]) -> ([Signal Bool],[Signal Bool])
dropP = id -|- odds
--

lastXor (as, bs) = map xor2 cs
   where
      cs = zipp (as, bs)
--

sums (a:as, bs) = (a: lastXor (as, init bs),carryOut)
   where
      carryOut = last bs
--

bKungFastAdder = gAndPs ->- fork ->- (evens -|- bKung) ->- dropP ->- sums

\end{lstlisting}
