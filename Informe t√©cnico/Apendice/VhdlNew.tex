\chapter{\textsc{ Librería de Lava para generar netlist VHDL}}\label{chap:vhdlnew-lava}
Librería modificada para generar un \cursi{netlist} VHDL apropiado para el Silicon Compiler de Electric. Se puede descargar la última versión de esta librería en \url{http://bit.ly/1vISP2r}

\lstset{language=Haskell,breaklines=true,extendedchars=true,mathescape=false} 
\lstinputlisting{Apendice/VhdlNew.hs}

%\begin{lstlisting}[mathescape=false]
%module VhdlNew
%  ( writeVhdlClk
%  , writeVhdlNoClk
%  , writeVhdlInputClk
%  , writeVhdlInputNoClk
%  , writeVhdlInputOutputClk
%  , writeVhdlInputOutputNoClk
%  )
% where

%import Signal
%import Netlist
%import Generic
%import Sequent
%import Error
%import LavaDir

%import List
%  ( intersperse
%  , nub
%  )

%import IO
%  ( openFile
%  , IOMode(..)
%  , hPutStr
%  , hClose
%  )

%import System.IO
%  ( stdout
%  , BufferMode (..)
%  , hSetBuffering
%  )

%import Data.IORef

%import IOBuffering
%  ( noBuffering
%  )

%--import IOExts
%--  ( IORef
%--  , newIORef
%--  , readIORef
%--  , writeIORef
%--  )

%import System
%  ( system
%  , ExitCode(..)
%  )

%----------------------------------------------------------------
%-- write vhdl

%writeVhdlClk :: (Constructive a, Generic b) => String -> (a -> b) -> IO ()
%writeVhdlClk = writeVhdl True

%writeVhdlNoClk :: (Constructive a, Generic b) => String -> (a -> b) -> IO ()
%writeVhdlNoClk = writeVhdl False

%writeVhdl :: (Constructive a, Generic b) => Bool -> String -> (a -> b) -> IO ()
%writeVhdl clocked name circ =
%  do writeVhdlInput clocked name circ (var "inp")

%writeVhdlInputClk :: (Generic a, Generic b) => String -> (a -> b) -> a -> IO ()
%writeVhdlInputClk = writeVhdlInput True

%writeVhdlInputNoClk :: (Generic a, Generic b) => String -> (a -> b) -> a -> IO ()
%writeVhdlInputNoClk = writeVhdlInput False

%writeVhdlInput :: (Generic a, Generic b) => Bool -> String -> (a -> b) -> a -> IO ()
%writeVhdlInput clocked name circ inp =
%  do writeVhdlInputOutput clocked name circ inp (symbolize "outp" (circ inp))

%writeVhdlInputOutputClk :: (Generic a, Generic b)
%                     => String -> (a -> b) -> a -> b -> IO ()
%writeVhdlInputOutputClk = writeVhdlInputOutput True

%writeVhdlInputOutputNoClk :: (Generic a, Generic b)
%                     => String -> (a -> b) -> a -> b -> IO ()
%writeVhdlInputOutputNoClk = writeVhdlInputOutput False


%writeVhdlInputOutput :: (Generic a, Generic b)
%                     => Bool -> String -> (a -> b) -> a -> b -> IO ()
%writeVhdlInputOutput clocked name circ inp out =
%  do writeItAll clocked name inp (circ inp) out

%writeItAll :: (Generic a, Generic b) => Bool -> String -> a -> b -> b -> IO ()
%writeItAll clocked name inp out out' =
%  do noBuffering
%     putStr ("Writing to file \"" ++ file ++ "\" ... ")
%     writeDefinitions clocked file name inp out out'
%     putStrLn "Done."
% where
%  file = name ++ ".vhd"
%  
%----------------------------------------------------------------
%-- definitions

%writeDefinitions :: (Generic a, Generic b)
%                 => Bool -> FilePath -> String -> a -> b -> b -> IO ()
%writeDefinitions clocked file name inp out out' =
%  do firstHandle  <- openFile firstFile WriteMode
%     secondHandle <- openFile secondFile WriteMode
%     var <- newIORef 0
%     

%     hPutStr firstHandle $ unlines $
%       [ "library ieee;"
%       , ""
%       , "use ieee.std_logic_1164.all;"
%       , ""
%       , "entity"
%       , "  " ++ name
%       , "is"
%       , "port"
%       , "  ( "
%       , if clocked then "    clk : in std_logic ;" else " "
%       , "    "] ++   -- , "  -- inputs"] ++
%       [ "    " ++ v ++ " : in std_logic" | VarBool v <- [head inps]] ++
%       [ "  ; " ++ v ++ " : in std_logic"
%       | VarBool v <- tail inps
%       ] ++
%       [ ""
%       , "  " -- outputs
%       ] ++
%       [ "  ; " ++ v ++ " : out std_logic"
%       | VarBool v <- outs'
%       ] ++
%       [ "  );"
%       , "end " ++ name ++ ";"
%       , ""
%       , "architecture"
%       , "  structural"
%       , "of"
%       , "  " ++ name
%       , "is"
%	, " --Agregado para que Electric encuentre las celdas estandards"
%	, "component and2"
%	, "port( A, B : in std_logic;  Y : out std_logic);"
%	," end component;"
%	, "component or2"
%	, "port( A, B : in std_logic;  Y : out std_logic);"
%	," end component;"
%	, "component xor2"
%	, "port( A, B : in std_logic;  Y : out std_logic);"
%	," end component;"
%	, "component id"
%	, "port( A : in std_logic;  Y : out std_logic);"
%	," end component;"
%	,"--"
%       ]

%       
%     hPutStr secondHandle $ unlines $
%       [ "begin"
%       ]
%     
%     let new =
%           do n <- readIORef var	
%              let n' = n+1; v = "w" ++ show n'
%              writeIORef var n'
%              hPutStr firstHandle ("  signal " ++ v ++ " : std_logic;\n")
%              return v
%         
%         define v s =
%           case s of
%             Bool True     -> port "vdd"  []
%             Bool False    -> port "gnd"  []
%             Inv x         -> port "inv"  [x]

%             And []        -> define v (Bool True)
%             And [x]       -> port "id"   [x]
%             And [x,y]     -> port "and2" [x,y]
%             And (x:xs)    -> define (w 0) (And xs)
%                           >> define v (And [x,w 0])

%             Or  []        -> define v (Bool False)
%             Or  [x]       -> port "id"   [x]
%             Or  [x,y]     -> port "or2"  [x,y]
%             Or  (x:xs)    -> define (w 0) (Or xs)
%                           >> define v (Or [x,w 0])

%             Xor  []       -> define v (Bool False)
%             Xor  [x]      -> port "id"   [x]
%             Xor  [x,y]    -> port "xor2" [x,y]
%             Xor  (x:xs)   -> define (w 0) (Or xs)
%                           >> define (w 1) (Inv (w 0))
%                           >> define (w 2) (And [x, w 1])
%                           
%                           >> define (w 3) (Inv x)
%                           >> define (w 4) (Xor xs)
%                           >> define (w 5) (And [w 3, w 4])
%                           >> define v     (Or [w 2, w 5])

%             VarBool s     -> port "id" [s]
%             DelayBool x y -> if clocked then port "delay" [x, y] else wrong Error.DelayEval
%             DlyBool x -> if clocked then port "dly" [x] else wrong Error.DlyEval
%             

%             _             -> wrong Error.NoArithmetic
%           where
%            w i = v ++ "[" ++ show i ++ "]"
%            


%            
%            port "delay" [x, y] =
%             do hPutStr secondHandle $
%                    "  "
%                   ++ make 9 ("c_" ++ v)
%--                   ++ " :  std_"
%                   ++ " :  "

%                   ++ make 5 "dff"
%                   ++ " port map ("
%                   ++ concat (intersperse ", " ("clk":[y] ++ [v]))
%                   ++ ");\n"   


%            

%            port name args =
%              do hPutStr secondHandle $
%                      "  "
%                   ++ make 9 ("c_" ++ v)
%--                   ++ " :  std_"
%                   ++ " :  "

%                   ++ make 5 name
%                   ++ " port map ("
%                   ++ concat (intersperse ", " (args ++ [v]))
%                   ++ ");\n"    

%     outvs <- netlistIO new define (struct out)
%     hPutStr secondHandle $ unlines $
%       [ ""
%       , "  " -- , "  -- naming outputs"
%       ]
%     
%     sequence
%       [ define v' (VarBool v)
%       | (v,v') <- flatten outvs `zip` [ v' | VarBool v' <- outs' ]
%       ]
%     
%     hPutStr secondHandle $ unlines $
%       [ "end structural;"
%       ]
%     
%     hClose firstHandle
%     hClose secondHandle
%     
%     system ("cat " ++ firstFile ++ " " ++ secondFile ++ " > " ++ file)
%--     cat firstFile secondFile file
%     system ("rm " ++ firstFile ++ " " ++ secondFile)
%    -- Crear un netlist VHDL sin la instancia de 'wire' o 'id' ñ ó:
%     system ("/home/lean/tfinal/programas/lava/la-va/Lava2000/Scripts/deleteWire.pl " ++ file)
%     return ()
% where
%  sigs x = map unsymbol . flatten . struct $ x
%  
%  inps  = sigs inp
%  outs' = sigs out'
% 
%  firstFile  = file ++ "-1"
%  secondFile = file ++ "-2"

%  make n s = take (n `max` length s) (s ++ repeat ' ')

%cat a b c = do
%  do s <- readFile a
%     writeFile c s
%  do s <- readFile b
%     appendFile c s

%----------------------------------------------------------------
%-- the end.

%\end{lstlisting}

