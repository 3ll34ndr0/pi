\chapter{ \textsc{ Introducción } }
En el presente capítulo se describe de forma general la estructura y contenido de todo el trabajo. Detallamos el planteo del problema y la motivación para llevarlo acabo. Luego definimos los objetivos generales y particulares, y enunciamos el plan de trabajo que nos hemos propuesto.
\section{Estructura del Proyecto Integrador}

\begin{description}
\item[Parte \ref{disenio_digital} - Diseño digital:] Definimos las métricas que utilizaremos para seleccionar calificar la implementación de los sumadores. Seleccionamos tres arquitecturas para implementar y desarrollamos los fundamentos teóricos de estas implementaciones. Seleccionamos un lenguaje de descripción de \cursi{hardware}, implementamos los circuitos, simulamos y verificamos formalmente los mismo. Generamos tres \netlist VHDL de cada arquitectura para 3 tamaños de sumandos distintos: 8, 16 y 32 bits.
\item[Parte \ref{disenio_fisico} - Implementación Física:] Explicamos el flujo de diseño físico, seleccionamos herramientas de software para realizarlo, seleccionamos una tecnología de fabricación, creamos nuestras celdas estándar y realizamos de forma automática el \layout del circuito. Simulamos los circuitos para obtener las mediciones de nuestras métricas de calidad y comparamos todas las implementaciones.
\item[Parte \ref{conclusiones} - Conclusiones:] Realizamos un resumen de lo que se logró, haciendo énfasis en la metodología, los resultados y las aplicaciones. Planteamos mejoras a nuestro trabajo y desafíos futuros para el diseño de circuitos integrados.
\end{description}

%Relación conceptual entre capítulos

\section{Planteamiento del problema y motivación}
En la actualidad los microprocesadores, DSP, microcontroladores, y otro hardware específico para cálculo computacional o de aplicación específica son desarrollados en tecnología CMOS submicrónica. El problema planteado es, ¿Cómo hacer para diseñar circuitos integrados en ésta tecnología, con herramientas flexibles, libres\footnote{En el sentido que no impongan restricciones de uso, estudio, mejora y distribución.} y accesibles para todo tipo de uso: académico y comercial?.

Nos interesa este problema por varios motivos, pero el principal es poder acercar estas tecnologías sin restricciones a los estudiantes de grado. Las posibilidades de aplicaciones de los circuitos integrados son infinitas, ya que podemos controlar todas las dimensiones físicas de los transistores y elementos pasivos que integramos. Utilizando software libre, también existe la posibilidad de adaptar o mejorar las herramientas que utilizamos para el diseño.

Otra motivación importante es la económica: Los costos por licencias de las herramientas de software, pueden transformar un proyecto de bajo volumen de producción en económicamente no factible. Damos un ejemplo real: 

\begin{quote}
El costo de 25 chips en tecnología de 0.35~\microm con un área máxima de 3\mmcuadrado, con encapsulado DIP-20 fabricados por medio de CMP\footnote{CMP es una organización que presta el servicio para la fabricación de circuitos integrados y MEMS para prototipado y bajo volumen de producción} es de \euro{2787.5}, resultando el costo de fabricación en \euro{111.5} por unidad. Pero si tenemos en cuenta el costo de las licencias de software por un año, el costo total se incrementa en un orden de magnitud, como mínimo\footnote{El costo de licencia de un conjunto completo de herramientas por un año puede llegar hasta \$1.000.000 (dólares norteamericanos). Este tipo de información no está públicamente disponible.}.
\end{quote}

En función del problema planteado y de las motivaciones mencionadas, nos propusimos implementar un sumador binario ya que es un circuito presente en la mayoría de los sistemas, que además puede ser el bloque de mayor requerimiento de performance.
\section{Objetivos generales}

El objetivo del proyecto es diseñar un sumador de n-bits, que pueda ser utilizado en un sistema digital, especificando las características más importantes que nos permitan determinar su utilización según requerimientos de tamaño en bits, performance, potencia y área. Lograremos un diseño paramétrico según la cantidad de bits, que podrá ser usado en unidades aritméticas o formar parte de un sistema de procesamiento de señales digitales. Integrar y documentar una metodología de diseño utilizando herramientas de software libre también es un objetivo de este trabajo, para que pueda ser reproducida, mejorada o adaptada a las necesidades de futuros proyectos que se planteen el diseño de circuitos integrados.


\section{Objetivos particulares}
El objetivo general de este proyecto integrador se puede lograr si nos planteamos los siguientes objetivos particulares:

\begin{itemize}
\item Estudiar las técnicas actuales para implementar sumadores rápidos y definir métricas de calidad.
\item Relevar las herramientas de software disponibles para hacer diseño digital y diseño físico, y las tecnologías de fabricación de circuitos integrados.
\item Implementar la suma con una arquitectura básica para referencia y comparación, y dos arquitecturas rápidas. Implementar los circuitos en un lenguaje de descripción de hardware y realizar la implementación física.
\item Crear \cursi{scripts} que nos permitan pasar de una etapa a la otra del diseño de forma automática, realizando las modificaciones necesarias para adaptar la salida de un proceso a la entrada del siguiente.
\item Verificar su funcionamiento y extraer métricas de calidad para poder compararlas.
\end{itemize}

\section{Plan de Trabajo}
Se estableció el siguiente plan de trabajo para llevar adelante el proyecto:
\begin{enumerate}
\item Comprender los requerimientos del diseño y fabricación de los circuitos integrados, y la relación con las herramientas de software.
\item Diseñar cada una de las arquitecturas a implementar, simularlas y verificar su correcto funcionamiento. 
\item Realizar bancos de prueba (\cursi{test benches}) para poder generar las métricas de calidad.
\item Generar conclusiones a partir de los resultados y proponer desafíos futuros.

\end{enumerate}


