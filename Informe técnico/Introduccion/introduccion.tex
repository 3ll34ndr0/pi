\chapter{ \textsc{ Introducción } }
\begin{abstract}
En el presente capítulo se describe en rasgos generales el flujo para el diseño de Circuitos Integrados de Aplicación Específica (\emph{ASIC} por su sigla en inglés), y la metodología utilizada para llevar adelante el diseño, implementación y tape out del mismo.
\end{abstract}
\section{Estructura del Proyecto Integrador}

\section{Planteamiento del problema y motivación}
El objetivo del trabajo es diseñar un sumador de n-bits, que pueda ser enviado a fabricar utilizando
procesos de fabricación CMOS para circuitos integrados. Integrar y documentar un flujo de diseño
de este sistema digital utilizando herramientas de Software Libre, será un subproducto de este
diseño, para lograr la base de conocimiento necesaria en el diseño de circuitos integrados con
tecnología CMOS. Este trabajo además de integrar todos los procesos de diseño de un Circuito
Integrado, pretende facilitar el acceso a las herramientas de diseño de circuitos integrados a los
estudiantes de grado.

\section{Objetivo}


\section{Plan de Trabajo}

