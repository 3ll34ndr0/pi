\chapter{ \textsc{ Introducción } }
\begin{abstract}
En el presente capítulo se describe en rasgos generales el flujo para el diseño de Circuitos Integrados de Aplicación Específica (\emph{ASIC} por su sigla en inglés), y la metodología utilizada para llevar adelante el diseño, implementación y tape out del mismo.
\end{abstract}
\section{Estructura del Proyecto Integrador}

Este proyecto cuenta con 4 partes: Las primeras tres partes:
 
\ref{seleccion_arquitectura}: Selección de la Arquitectura \\
\ref{implementacion_fisica}: Implementación Física \\
\ref{comparacion_resultados}: Comparación de resultados. 

Estas tres partes forman un flujo de trabajo circular e iterativo.

Y finaliza con un resumen de las conclusiones, además de conjunto de anexos técnicos ubicados al final para su consulta.


\section{Planteamiento del problema y motivación}
En la actualidad los microprocesadores, los DSP, los microcontroladores, y otro hardware específico para cálculo computacional son desarrollados en tecnología CMOS submicrónica. El problema planteado es, ¿Cómo hacer para diseñar circuitos integrados en esta tecnología, con herramientas flexibles, libres\footnote{En el sentido que no impongan restricciones de uso, estudio, mejora y distribución.} y accesibles para todo tipo de uso: academico y comercial?.

\section{Objetivo}

El objetivo del trabajo es diseñar un sumador de n-bits, por ser este el elemento central de cualquier tipo de circuito digital de cálculo: Los multiplicadores, los MAC (\emph{multiply-accumulate}), los filtros FIR, etc, que pueda ser enviado a fabricar utilizando
procesos de fabricación CMOS para circuitos integrados. Integrar y documentar un flujo de diseño
de este sistema digital utilizando herramientas de Software Libre, será un subproducto de este
diseño, para lograr la base de conocimiento necesaria en el diseño de circuitos integrados con
tecnología CMOS. Este trabajo además de integrar todos los procesos de diseño de un Circuito
Integrado, pretende facilitar el acceso a las herramientas de diseño de circuitos integrados a los estudiantes de grado.
\section{Plan de Trabajo}

