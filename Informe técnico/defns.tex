%http://mirror.ox.ac.uk/sites/ctan.org/macros/latex/contrib/glossaries/glossaries-user.html#sec:printglossary
%makeindex -s informe.ist -o informe.gls informe.glo


\newglossaryentry{ex}{name={sample},description={an example}}
%\setacronymstyle{long-short}
%\setacronymstyle{long-sc-short}
\newacronym{lvs}{LVS}{Layout Vs. Schematic}
\newacronym{pnr}{PnR}{Place \& Route}
\newglossaryentry{drc}{name={DRC}, description={Design Rule Check. Verificación de las dimensiones físicas del \cursi{layout} para evitar violaciones a las reglas que el proceso de fabricación del chip impone, para evitar efectos no deseados (corto circuitos, tensiones internas del chip que lo puedan quebrar, etc)}}
\newacronym{pdk}{PDK}{Process Design Kit}
%%PDKs for Analog/Mixed-Signal/RF Design

%A Process Design Kit (PDK) is a collection of foundry-specific data files and script files used with EDA tools in a chip design flow. A PDK’s main components are models, symbols, technology files, parameterized cells (PCells), and rule files. With a PDK, designers can jump-start chip design and work through the design flow seamlessly, from schematic entry to tapeout.


\newacronym{cts}{CTS}{Clock Tree Synthesis}
\newacronym{idms}{IDMs}{Integrated Device Manufacturers}
\newglossaryentry{sta}{name={STA},description={Static Timing Analisys. Metodología para realizar estimaciones sobre los tiempos y potencia de los circuitos digitales, que puede lograr una precisión cercana a las simulaciones \gls{spice}, pero con mucho menos tiempo de cálculo. El estándard más utilizado es el formato abierto Liberty (\url{http://opensourceliberty.org/})}}
\newacronym{hdl}{HDL}{Hardware Description Language}
\newglossaryentry{cmp}{name={CMP},description={Organización dedicada a la fabricación de circuitos integrados y \gls{mems} para prototipado y bajo volumen de producción. Sitio web: \url{http://cmp.imag.fr/}}}

\newglossaryentry{padframe}{name={Pad frame},description={Es un conjunto de celdas que se ubican en el marco del \gls{die}, para conectar las señales del circuito con el exterior del chip}}

\newglossaryentry{die}{name={Die},description={Die o Chip, es una pieza de material semiconductor (mayormente silicio) en la cual se construyen todos los transistores, resistores, capacitores e inductores por medio de procesos físico químicos}}

\newglossaryentry{mems}{name={MEMS},description={Sistemas Micro-Electro-Mecánicos, por sus sigas en inglés}}

%\newacronym{vhdl}{VHDL}{\gls{vhsic} Hardware Description Language}
\newglossaryentry{vhdl}{name={VHDL},description={\gls{vhsic} Hardware Description Language}}

\newacronym{asic}{ASIC}{Aplication Specific Integrated Circuit}
\newglossaryentry{vhsic}{name={VHSIC},description={Very High Speed Integrated Circuit}}
\newglossaryentry{spice}{name={SPICE},description={Simulation Program with Integrated Circuit Emphasis. Motor de simulación a nivel de transistores y elementos pasivos. Desarrollado por la universidad de Berkley y liberado con licencia libre, se transformó en el estándard de facto para la simulación de circuitos analógicos. Partiendo de la versión 3f5 se desarrollaron otras varias implentaciones, siendo las más conocidas PSPICE, HSPICE, spectre, ngspice, LTSPICE, entre otros. Cada implementación no es compatible (por su sintáxis o por el tipo de análisis que soporta) con las otras, aunque dependiendo del caso se puede salvar con pequeñas modificaciones}}
\newglossaryentry{cla}{name={CLA},description={Carry Lookahead Adder}}
\newglossaryentry{vcd}{name={VCD},description={Value Change Dump. Es un formato basado en ASCII para loguear señales, utilizado por herramietas de simulación lógica. Para visualizarlo podemos utilizar el software GTKWave, de licencia libre
}}
%\newacronym{vcd}{VCD}{Value Change Dump}
\newglossaryentry{sat}{name={SAT},description={Boolean satisfiability problem}}
\newglossaryentry{tsmc}{name={TSMC},description={Taiwan Semiconductor Manufacturing Company}}
